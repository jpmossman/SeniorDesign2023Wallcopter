% Import Packages
\documentclass[12pt]{article}
\usepackage[utf8]{inputenc}
\usepackage{xcolor}
\usepackage{listings}
\usepackage[letterpaper, margin=1in]{geometry}
\usepackage{graphicx}
\usepackage{verbatim}
\usepackage{underscore}
\usepackage{fancyhdr}
\usepackage{hyperref}
\usepackage{enumitem}
\usepackage{hanging}
\usepackage{seqsplit}
\usepackage{multirow}
\usepackage{multicol}
\usepackage{array}
\usepackage{longtable}
\usepackage{xurl}

% Command for quick and easy bibliography
\newcommand\hangingindent[1]{\begin{hangparas}{0.5in}{1} #1 \end{hangparas}\vspace{0.5cm}}

% Change the color of hyperlinks to blue
\renewcommand\UrlFont{\color{blue}}

\pagestyle{fancy}
\fancyhf{}
\fancyhead[R]{Wall Guys: Assignment 1}
\fancyhead[L]{\leftmark}
\fancyfoot[C]{\thepage}

\begin{document}

    \begin{titlepage}
        \begin{center}
            \vspace*{1cm}
            \Huge\textbf{ECE 396: Assignment 1}

                \vspace{0.5cm}
                \LARGE Project Selection \& Problem Scoping
    
            \vspace{1.5cm}
            \textbf{Team 13: Wall Guys}

                \vspace{0.5cm}
                Tien Dao, Hardik Goel, Jaden Mossman, and Anand Pudi

            \vfill             
            \includegraphics[width=0.4\textwidth]{resources/uic_logo.png}

            % ASSIGNMENT DESCRIPTION
            \vspace{0.8cm}
            University of Illinois Chicago\\
            Fall 2022
                
        \end{center}
    \end{titlepage}

    \tableofcontents

    \newpage

    \section{Overview}

        \subsection{Project Goals}
            The aim of this project is to develop a drone that can touch or get very close to the bottom side of a bridge using its wheels. We are looking to make at least a prototype that can go up to the ceiling and can navigate using a wireless control system. We will be prioritizing functionality, trying to get a minimum working prototype, and then try to reduce the cost of the drone itself if possible.

        \subsection{Needs Statement}
            High rise buildings and the undersides of bridges are important structures that need to be examined for safety purposes, but this is often a difficult task as they are hard to reach places and typically windy and tough environments. Drones are often deployed to conduct these examinations, but there limitations to their performance, such as “From the perspective of safety operations, the multi-rotor drone should be kept away from both the upper wall and the side wall at a distance of at least 1.5 times of the rotor diameter to prevent unexpected motions of the aircraft caused by the wall during hovering flight.” (Tanabe 344). These limitations, among others, provide a challenge to the autonomous examination of bridges and buildings.

        \subsection{Objective Statement}
            In order to solve the proposed problem, we will be using a set of wheels at the top of the drone that allows it to closely traverse the ceiling. A set of sensors will be used to estimate the distance to the ceiling, which will be utilized in a feedback loop to control the distance from the ceiling and autonomously rise to it.

        \newpage
        \subsection{Background}
            \begin{enumerate}[label=\Alph*.]
                \item The basic concept that lays behind this wall-copter relies on the control system that keeps the copter stable to prevent collision or unintended side effect when approaching ceilings or walls. The current limitation lays on the unintended shifts of the blade speed when the copter gets too close to the wall's surface. The other factor is the difficulty of control when the drone gets too close to such a surface. Surprisingly, there don't seem to be any current remedies that exist in the market right now so we do not have a way to compare our project to it. This, in turn, allows us to have a fairly unique product without competitors. One thing to note is that this runs with the assumptions of there being no other similar projects under research. To our knowledge and research, there seems to be no patents that exist of this product that we are trying to develop, strange as it seems to be. This allows us to have an open field with no current competitors.
                
                Articles that relates to our topic (no patent due to the reason explained above): 
                \begin{itemize}[label=]
                    \item The effects of walls on motors:\\\url{https://www-fujipress-jp.proxy.cc.uic.edu/jrm/rb/robot003000030344/}
                    \item Experimentation on a wall climbing drone:\\\url{https://ieeexplore-ieee-org.proxy.cc.uic.edu/document/8588997}
                \end{itemize}

                \item Our background of this project is very clear from the beginning since it's a sponsored project, we still interviewed the professor to answer more questions we had. From this, we understand that the professor wants a product that currently does not exist on the market and will be answering the demand that some have for drones.
                \begin{enumerate}[label=\roman*.]
                    \item Functionality: The goal is to make a drone which navigates in high places without the current problems (shaking, height limit close to ceiling, stability).
                    \item Appearance: It will look just like a common drone that currently exists in the market except with add-ons that would assist it with our goal.
                    \item Cost: The cost of the drone is preferred to be as low as possible since it is a commercial product. For now our goal is just your standard drone cost that one can find for a reasonable price range of 40-100 dollars.
                    \item User Interface: Our user interface would be that of your standard drone control, but with potentially an extra set of control for the sets of wheels that will be at the top of the device.
                    \item Reliability: We want to obtain a product that will be consistent with every use. This device should be very reliable since it builds upon the existing frame of a normal drone.
                    \item Power Requirements: The drone runs on a standard rechargeable battery which can also power our wheel system. Perhaps additional modification might be required if there is more power required to run the drone, but currently it would be safe to assume a standard drone battery would be able to run just fine also.
                    \item Expected Product Lifetime: As this kind of drone would be more difficult to navigate, we can assume it has a lower average product lifetime. However, with proper use and care, it should still last only a little bit less if not the same amount of time as a standard drone.
                    \item Interfacing requirements with other systems.
                    \item User training needs: It is safe to assume that there will be a learning curve to the device in addition to learning how to navigate using the extra set of wheels on top. Such a process can be learned just like a normal drone and should still be able to learn with some time and effort.
                    \item Government regulations and licensing: According to Part 107 of the FAA, any drone that weights over 250g needs to be registered even if it's only for commercial purposes. This process shouldn't be too difficult and there isn't too much of a barrier to obtain one for use.
                    \item Industry standards: We follow the rules given to all drones. Since our product is newer, we don't know what other standards might be applied to it. So currently we'll be abiding by the current drone industry standards.
                    \item Safety issues: The drone has the issue of crashing into high terrains and falling onto passerbys below. There is also potential that objects that the wheels made contact with on the ceiling could potentially break and fall. The device has the same safety problems as normal drones, only with additional unforeseen safety issues that come with driving it on the wall/ceiling.
                    \item Environmental Concerns: This device should be fairly safe environmentally to operate, with a low carbon footprint. There are still concerns that when it crashes or is lost, if not recovered, it would be non biodegradable as it is made out of plastics and other electronic components. The battery that is used to operate the drone is also bad for the environment if not recycled correctly.
                \end{enumerate}
            \end{enumerate}
        \newpage

        \subsection{Marketing Requirements}
            \begin{enumerate}[label=\arabic*.]
                \item The wall-copter must be able to climb on flat surfaces.
                \item The wall-copter should be stable.
                \item The wall-copter should be easy to use.
                \item The wall-copter should be cost effective.
                \item The wall-copter must be able to be controlled by a user from a distance.
            \end{enumerate}
        
        \subsection{Objective Tree}
            \centerline{\Large \underline{\textbf{Wall Copter}}}
            \begin{multicols}{3}
                \noindent\underline{Climb on Flat Surfaces 0.45}
                
                \noindent Walls 0.10
                
                \noindent Ceilings 0.70
                
                \noindent Underneath bridges 0.20

                \columnbreak

                \noindent\underline{Easy to Use 0.10}

                \noindent Limited controls 0.36

                \noindent Long operating time 0.64

                \columnbreak
                
                \noindent\underline{Be Stable 0.45}
                
                \noindent Sturdy structure 0.42
                
                \noindent Accurate motion sensors 0.35
                
                \noindent Long-lasting motors 0.23

            \end{multicols}
    
    \newpage
    \section{Engineering Requirements}
        \begin{longtable}{ | m{4.25cm} | m{5cm}| m{5.4cm} | }
            \hline
            Marketing Requirement & Engineering Requirement & Justification \\\hline
            1. Functionality &
            The wall-copter should be able to go to the ceiling and also maneuver at the top. &
            The product should adhere to the drone standard of being airworthiness (IEEE 1936.1-2021) \\\hline
            2. Performance &
            The wall-copter should have standard drone speed and also travel consistently using the wheels when at the top. &
            The device should be able to meet mechanical, and electrical requirements (IEEE 1937.1-2020) \\\hline
            3. Economic &
            The device should be within a reasonable price range. &
            Although there isn't a requirement for prices, the device should follow standards for estimating costs (IEEE 3001.4-2020)\\\hline
            4. Energy/Power &
            The wall-copter should be able to run for at least 1 hour and is rechargeable. &
            The battery should follow IEEE guidelines for safety and usability (IEEE 1625-2008)\\\hline
            5. Environmental &
            The device should not emit any pollution and should not affect its surrounding environment too much. &
            The device should be following all the standards and remain environmentally safe (IEEE 1680.1) \\\hline
            6. Health \& Safety &
            The user should follow proper instruction in order to protect the wellbeing of themselves and others around them. &
            The user needs to be properly taught to operate the wall-copter in order to be safe for themselves and others (IEEE 1936.1-2021) \\\hline
            7. Legal &
            The user needs to have the proper license needed in order to operate the device. &
            Qualification of operator is needed in order to use the product (IEEE 1936.1-2021) \\\hline
            8. Maintainability &
            The wall-copter should be easy to clean and replace parts on. &
            The device have to adhere to proper functionality for easy maintenance (IEEE 1937.1-2020) \\\hline
            9. Manufacturability &
            The wall-copter should be easy to made with additional modification to a standard drone production. &
            The product needs to be in line with manufacturing process control (IEEE 1625-2008) \\\hline
            10. Reliability \& Availability &
            The wall-copter should be able to perform all tasks along with being able to be purchased readily on the market. &
            The device should be very reliable in order for repeated use (IEEE 1680.1) \\\hline
            11. Usability (including user interface) &
            The user should be able to operate the device after some basic training similar to how a normal drone should operate. &
            As the product require data to be collected for operations, it should also follow the drone data classification (IEEE 1936.1-2021) \\\hline
        \end{longtable}
    
    \newpage
    \section{Engineering Design Alternatives}
        \subsection{Differences and Similarities Between Designs}
            Our three designs are very similar in principle as they all have the goal of detecting when the drone is close to the ceiling and giving feedback to the system. The designs were built upon this principle, keeping the drone from hitting the ceiling. They also have ways in which they navigate while touching the ceiling. They all share a control system in which we communicate to it through a controller, what we usually see in standard drones. Although these designs are similar, they are quite different as we try to create variations that can hopefully determine the best outcome. Their main differences are the sensors that go on them. Design 1 uses an ultrasonic sensor while Design 2 uses a camera with imaging processing and Design 3 uses lasers. Each of these variations have their own strengths and weaknesses that vary with costs. However, they all achieve the same goal of telling the drone if there's an object overtop of it (and how close) in order to avoid collision. Another difference is how the drone is going to navigate over the ceiling, with Design 1 and Design 3 both using ball bearing wheels that would just serve as a way to move. In that case, movement will be handled by the drone itself, reducing the design complexity. Design 2 uses motorized wheels, which runs using its own motor, this will allow the drone to move without compromising its vertical movements. One last difference is how they are controlled, all would have the same communications via WiFi with the main difference being that Design 1 and Design 3 using a standard drone controller, while Design 2 aims to use a phone app as its control.

        \newpage
        \subsection{Design 1}
            This design includes an ultrasonic sensor in order to determine the distance between the drone and the ceiling/objects. We would have ball bearing wheels at the top in various selected locations. The drone would handle the movement both vertically and horizontally. In order to control the drone, we would be using a controller that communicates with the drone. The drone still would include a camera on top for image capturing.

            \vspace{0.5in}
            \centerline{\includegraphics{./resources/assignment3-Design1.schematic.drawio.pdf}}

            \vspace{0.5in}
            \begin{tabular}{|c|l|}
                \hline
                \multicolumn{2}{|c|}{\textbf{Design 1, Level 0}} \\\hline
                Module & Wall-Copter \\\hline
                \multirow{3}{3cm}{Inputs}
                    & - Power: Lithium Ion Battery\\
                    & - User Input by Controller \\
                    & - Sensor Data (Ultrasonic Sensor) \\
                    \hline
                \multirow{2}{3cm}{Outputs}
                    & - Drone Movement \\
                    & - Image Data Capturing \\
                    \hline
                \multirow{3}{3cm}{Functionality}
                    & - Precision flight control close to the roof through sensor data. \\
                    & - Movement along the roof with simple ball bearing wheels. \\
                    & - Capturing image for building inspections. \\
                    \hline
                
            \end{tabular}
            
        \newpage
        \subsection{Design 2}
            Using the idea of having two cameras (one on top, one built in) with an imaging system that would be used to determine the distance to the wall. Will use motorized wheels on top of the drone for horizontal navigation.
            
            \vspace{0.5in}
            \centerline{\includegraphics{./resources/assignment3-Design2.schematic.drawio.pdf}}

            \vspace{0.5in}
            \begin{tabular}{|c|l|}
                \hline
                \multicolumn{2}{|c|}{\textbf{Design 1, Level 0}} \\\hline
                Module & Wall-Copter \\\hline
                \multirow{3}{3cm}{Inputs}
                    & - Power: Lithium Ion Battery\\
                    & - User Input by Phone App \\
                    & - Imaging Data \\
                    \hline
                \multirow{3}{3cm}{Outputs}
                    & - Drone Movement \\
                    & - Wheel Movement \\
                    & - Image Data \& Video \\
                    \hline
                \multirow{4}{3cm}{Functionality}
                    & - Precision flight control close to the roof. \\
                    & - Capturing images and using them for computer vision flight control. \\
                    & - Live feedback of data through WiFi. \\
                    & - Navigation on the roof through motorized wheels. \\
                    \hline
                
            \end{tabular}

        \newpage
        \subsection{Design 3}
            This design uses lasers placed in triangular configuration to triangulate the correct distance between the drone and the ceiling. It will also use ball bearing wheels as a way safely and efficiently navigate the ceiling. The drone will also be controlled by a standard controller that comes with the product.

            \vspace{0.5in}
            \centerline{\includegraphics{./resources/assignment3-Design3.schematic.drawio.pdf}}

            \vspace{0.5in}
            \begin{tabular}{|c|l|}
                \hline
                \multicolumn{2}{|c|}{\textbf{Design 1, Level 0}} \\\hline
                Module & Wall-Copter \\\hline
                \multirow{3}{3cm}{Inputs}
                    & - Power: Lithium Ion Battery\\
                    & - User Input by Controller \\
                    & - Laser Sensors \\
                    \hline
                \multirow{2}{3cm}{Outputs}
                    & - Drone Movement \\
                    & - Image Data Capturing \\
                    \hline
                \multirow{3}{3cm}{Functionality}
                    & - Using sensor data for stabilized flight control close to the roof. \\
                    & - Capturing image for building inspections. \\
                    & - Using ball bearing wheels to maneuver close/on the roof. \\
                    \hline
                
            \end{tabular}

    \newpage
    \section{Bibliography}
    \centerline{\large \underline{\textbf{Works Cited}}} \vspace{0.5cm}
        % TODO: Switch this to using biber

        \hangingindent{Tanabe, Sugiura, M., Aoyama, T., Sugawara, H., Sunada, S., Yonezawa, K., \& Tokutake, H. (2018). Multiple Rotors Hovering Near an Upper or a Side Wall. Journal of Robotics and Mechatronics, 30(3), 344-353. \seqsplit{https://doi.org/10.20965/jrm.2018.p0344}}

        \hangingindent{Myeong, \& Myung, H. (2019). Development of a Wall-Climbing Drone Capable of Vertical Soft Landing Using a Tilt-Rotor Mechanism. IEEE Access, 7, 4868-4879. \seqsplit{https://doi.org/10.1109/ACCESS.2018.2889686}}

        \hangingindent{Mattar, \& Kalai, R. (2018). Development of a Wall-Sticking Drone for Non-Destructive Ultrasonic and Corrosion Testing. Drones (Basel), 2(1), 8-. \seqsplit{https://doi.org/10.3390/drones2010008}}

        \hangingindent{Guo, Zhang, J., Ju, Y., \& Guo, X. (2018). Climbing Reconnaissance Drone Design. IOP Conference Series. Materials Science and Engineering, 452(4), 42060-. \seqsplit{https://doi.org/10.1088/1757-899X/452/4/042060}}

        \hangingindent{Become a drone pilot. Become a Drone Pilot | Federal Aviation Administration. (n.d.). Retrieved October 2, 2022, from \seqsplit{https://www.faa.gov/uas/commercial_operators/become_a_drone_pilot\#:~:text=In\%20order\%20to\%20fly\%20your,procedures\%20for\%20safely\%20flying\%20drones.}}

        \hangingindent{Karanja, P. (2022, February 24). Guide to how much Drones Cost (2022). Droneblog. Retrieved October 2, 2022, from \seqsplit{https://www.droneblog.com/drones-cost/\#:~:text=The\%20average\%20cost\%20of\%20drones,need\%20to\%20spend\%20on\%20it.}}
    
    \newpage
    \section{Appendices}
        \subsection{Appendix A}
            \subsubsection{Questions and Paraphrased Answers with Project Sponsor Professor Cetin}
            % \textbf{\large Questions and Paraphrased Answers with Project Sponsor Professor Cetin}
                \newcommand{\qna}[2]{
                    \noindent Q. #1

                    A. #2
                    
                    \vspace{0.5cm}

                }

                \qna
                {When it says two motors, does that include motors for steering? i.e. a cheap servo motor to point to a primary motor?}
                {Two motors as in two separate movement systems, movement and navigation}
                \qna
                {Is the drone expected to fly to the roof or wall ? Or is it supposed to navigate through the wall to the roof?}
                {To the roof.}
                \qna
                {Is there an ideal budget range?}
                {As cheap as possible, the University has a budget.}
                \qna
                {What is the expected size range? As small as functionally possible? Breadbox? 1m cube?}
                {Low cost -\> Small}
                \qna
                {Are there expected use cases aside from bridge and high-rise building inspection?}
                {Not yet considered, similar products do not yet exist.}
                \qna
                {If this is being used for inspections it presumably should have a camera, correct?}
                {Yes.}
                \qna
                {Other than the ones listed, what other type of terrains did you have in mind when you made this project?}
                {}
                \qna
                {What kind of model do you have in mind? Since there are some specifications listed for the wheels amount and propeller types.}
                {Similar to DJI Avata Pro-View}
                \qna
                {What kind of consumer/user did you have in mind when you came up with the project idea?}
                {Not yet considered, similar products do not yet exist.}
                \qna
                {Who is the intended user other than bridge and high-rise inspectors?}
                {Not yet known. Maybe people who want to clean the ceiling of large auditoriums.}
                \qna
                {What do you anticipate to be the biggest challenge to overcome?}
                {The control system required in order to keep the drone balanced once it touches the ceiling. More is required to prevent the drone from crashing.}
                \qna
                {How much deviation from the original prompt is acceptable? We are mainly concerned with the number of motors, but just in general}
                {Whenever we have a design in mind, we can check with the professor for approval of the design.}
                \qna
                {With regards to testing and demonstration, are there any recommended environments? Will any licenses be required? Do you know what the university's policy on using the Rec Center's rockwall as a control environment would be?}
                {The ECE labs and makerspace}
                \qna
                {What is your specialty and how does that relate to this?}
                {Image processing and if we can get a camera working that would be great}

                \noindent Extra info:

                If this gets good enough to be patented: https://otm.uic.edu/uic-community/how-to-disclose/

                Recommended courses: ece uic control
            
            % \vspace{1cm}\noindent\textbf{\large Original Sponsored Project Prompt}
            \subsubsection{Original Sponsored Project Prompt}

                \noindent\textbf{\large Project Title:}

                Wall-copter: Drone with wheels on its top that can navigate on the ceiling or walls.

                \noindent\textbf{\large Project Sponsor(s) Name and Email:}

                Ahmet Enis Cetin, aecyy@uic.edu

                \noindent\textbf{\large Description of Project:}

                Drones are used for bridge inspection and high-rise building inspection. However, they shake in the air and wind and camera and sensor measurements shake because of shaking. The aim of this project is to develop a drone that can touch or get very close to the bottom side of the bridge using its wheels.

                The propellors of the drone will push the drone up and the drone will use its three or four wheels to wonder on the wall, high ceiling or at the bottom side of a bridge. The drone can have two motors. One of the engines will power the propellers the other engine will power the wheels. You should use a single controller to control the system.

                You may be required to sign an NDA or relinquish IP rights for this project.
        
        \newpage
        \subsection{Appendix B}
            \newcommand{\ieeestd}[3]{
                \noindent\textbf{#1} \\
                \noindent\url{#2} \\
                \noindent#3 \\
                \vspace{0.5cm}

            }

            \ieeestd
            {IEEE 1936.1-2021 IEEE Standard for Drone Applications Framework}
            {https://standards.ieee.org/ieee/1936.1}
            {The drone safety and management requirements include airworthiness, airspace and air traffic requirements, qualification of operators, qualification of personnel, insurance, confidentiality, and others. The general operation process is detailed.}
            \ieeestd
            {IEEE 1625-2008 IEEE Standard for Rechargeable Batteries for Multi-Cell Mobile Computing Devices}
            {https://standards.ieee.org/ieee/1625}
            {This standard establishes criteria for design analysis for qualification, quality, and reliability of rechargeable battery systems for multi-cell mobile computing devices. The following are addressed: qualification process; manufacturing process control; energy capacity and demand management; levels of management and control in the battery systems; and current and planned lithium-based battery chemistries, packaging technologies, and considerations for end-user notification.}
            \ieeestd
            {IEEE 1937.1-2020 IEEE Standard Interface Requirements and Performance Characteristics of Payload Devices in Drones}
            {https://standards.ieee.org/ieee/1937.1/7456/}
            {General interface requirements and performance characteristics of payload devices in drones are presented. The drone payload interfaces are described in three categories: mechanical interface, electrical interface, and data interface.}
            \ieeestd
            {IEEE 3001.4-2020 IEEE Recommended Practice for Estimating the Costs of Industrial and Commercial Power Systems}
            {https://standards.ieee.org/ieee/3001.4/6808/}
            {Described in this recommended practice are methods for estimating the costs of industrial and commercial power systems, both new and those undergoing expansion or modernization. This recommended practice is restricted to the development of the relative capital cost of industrial and commercial power distribution systems.}
            \ieeestd
            {IEEE 1680.1 - Environmental Assessment of Computers, Tablets and Monitors}
            {https://rohs.ca/news/2018/04/16/ieee-1680-1-environmental-assessment-of-computers-tablets-and-monitors/}
            {The requirements take into account: substance management, materials selection, design for end of life, product longevity/life-cycle extension, energy conservation, end-of-life management, packaging, life cycle assessment and carbon footprint, corporate environmental performance, and corporate social responsibility.}
        
        \newpage
        \subsection{Appendix C}
            \subsubsection{General Concept Map}
                \centerline{\includegraphics{./resources/assignment3-general_concept_map.drawio.pdf}}
            
            \subsubsection{Alternative Design 1 Concept Map}
                \centerline{\includegraphics{./resources/assignment3-Design1.concept_map.drawio.pdf}}
            
            \subsubsection{Alternative Design 2 Concept Map}
                \centerline{\includegraphics{./resources/assignment3-Design2.concept_map.drawio.pdf}}
            
            \subsubsection{Alternative Design 3 Concept Map}
                \centerline{\includegraphics{./resources/assignment3-Design3.concept_map.drawio.pdf}}
        
\end{document}
